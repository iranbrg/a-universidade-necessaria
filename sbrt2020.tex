\documentclass{sbrt}
\usepackage[english,brazil]{babel}
\usepackage[utf8]{inputenc}

\begin{document}

\title{Sobre A Universidade Necessária}

\author{Iran Braga dos Santos Filho}

\maketitle

\begin{resumo}
Este documento apresenta um exemplo de utilização do estilo \LaTeX\ sbrt.cls para preparar um artigo para submissão ao SBrT~2020. O resumo deve conter no máximo 100 palavras.
\end{resumo}
\begin{chave}
Modelo de artigo, \LaTeX, SBrT~2020.
\end{chave}

\begin{abstract}
This document is an example of how to use the \LaTeX\ style sbrt.cls to prepare a paper for submission to SBrT~2020. The abstract must have at most 100 words.
\end{abstract}
\begin{keywords}
Paper template, \LaTeX, SBrT~2020.
\end{keywords}

\section{Introdução}

Elaborado pelo antropólogo, sociólogo, escritor e político Darcy Ribeiro, o projeto "A Universidade Necessária"{} foi um plano acadêmico, educacional e social. Nele, Darcy traçava um panorama dos dilemas e desafios cruciais com os quais a universidade, não só no Brasil, mas também em toda América Latina, se defrontava.

\section{O Projeto (Neo)Colonial de Educação}

O sistema de ensino no Brasil mostra-se deficiente desde os seus primórdios na fase colonial, em que o processo educacional concentrava-se nas mãos dos jesuítas - interessados na catequese, em virtude da Contrareforma -, no século XVI, e só apresentou alguma evolução com instauração de cursos de ensino superior após a chegada da família real, no século XIX.

Ainda assim, era claro o plano de subordinação de diferentes setores da elite metropolitana - comprometida em boicotar qualquer possibilidade de desenvolvimento da educação no Brasil. O propósito era não permitir que os laços da submissão do Brasil frente a Portugal fossem desatados. Essa subjugação intelectual perdurou durante tanto tempo que enraizou-se na cultura nacional. Hoje, essa prática colonialista é reeditada pelos países desenvolvidos, os quais são hegemônicos no cenário científico internacional, por tenderem a inibir o progresso acadêmico de países subdesenvolvidos ou em desenvolvimento.

Na segunda metade do século XX, não obstante, ganha notoriedade a figura de Darcy Ribeiro que reclama a incapacidade do domínio da ciência e das tecnologias, no Brasil, em um período marcado pela revolução científico tecnológica e informacional. Nesse contexto, o país mantinha-se nos níveis mais baixos da divisão internacional do trabalho no capitalismo.

Para Darcy Ribeiro, a nação teria de enfrentar severos desafios para superar o atraso de seu subdesenvolvimento - que também submete a América Latina, em geral - e sair do cenário de subserviência no sistema mundial. Darcy salientou, o problema do colonialismo e do neocolonialismo como centrais na explicação da presente situação do país.

\section{A Universidade Necessária}

Darcy defendia fortemente a autonomia das universidades. Para ele, o autogoverno delas deveria ser democrático e exercido pelos corpos acadêmicos, cuja política de ensino, pesquisa e extensão não poderia sofrer ingerências do Estado ou de organismos internacionais. Além disso, sabia que, para elas, naquele momento, seria mais fácil assumir a postura passiva de reproduzir as características de uma sociedade superficialmente moderna do que uma postura ativa de imprimir alterações que conduzissem a um autêntico desenvolvimento.

Por conseguinte, Darcy Ribeiro avaliou que, no ritmo de crescimento delas, não teriam possibilidades de desempenharem as funções mínimas de órgãos autônomos. Porém, se alterado o modelo das universidades, estariam elas aptas a anteciparem das transformações do contexto social, podendo nele intervir e provocar inovações intencionais. Nesse caso, a instituição universitária poderia ser o agente de transformação.

Em sua obra A Universidade Necessária, Darcy discorre sobre o projeto da UnB, onde foi reitor, e expõe sua proposta de universidade enquanto veículo de transformação da sociedadade e consequentemente da nação. Havia pontos específicos a serem enfrentados pela “universidade necessária” no país, dentre eles, o elitismo na educação, observado no estreitamento progressivo da oferta de matrículas no sistema público e gratuito junto à expansão desmedida do ensino privado de nível precário, incentivada, subretudo, pelo regime militar - o qual foi um grande opositor das ideias de Darcy Ribeiro.

No projeto da UnB, era enaltecido seu compromisso com a capacitação em alto nível dos estudantes, somando-se a isso o empenho em sua formação cidadã. Visava-se a formação integral, na unidade entre teoria e prática, de encontro ao modelo vigente de fragmentação do ensino provocado pela existência de inúmeras disciplinas apresentadas como isoladas e autossuficientes. O aprendizado era realizado sob a forma de treinamento em serviço, no atendimento das demandas reais, isto é, o aluno aprendia fazendo, característica essa também encontrada nos ideais de Anísio Teixeira, seu grande amigo, também idealizador de grandes mudanças no ensino público brasileiro.

No entanto, não se inovariam cursos, programas, currículos, se a universidade não juntasse intelectuais e cientistas de alto nível aos melhores talentos em início de carreira em condições promissoras de trabalho, dispondo da liberdade acadêmica, aguçados em sua imaginação e originalidade para dar respostas competentes aos desafios de seu tempo. Concomitantemente, havia de existir uma comunidade disposta a pensar uma inédita estruturação da universidade em institutos centrais e faculdades integrados.

% Era pressuposto que cabia à universidade pública e democrática atender à ampla e plural demanda por Ensino Superior no Brasil. Por isso, chama atenção, o cuidado de Darcy Ribeiro em reconhecer que os estudantes possuem expectativas diferentes em face da universidade, as quais podem mudar ou se consolidar ao longo dos anos. Em “A Universidade Necessária” (RIBEIRO, 1975b), vemos propostas três tipificações: a) consumidor; b) profissionalista; c) acadêmico. Este último é por ele dividido em dois subtipos: 1) técnico-profissional; 2) universitário. Tais denominações, talvez, fossem apreciadas nos anos em foram formuladas; hoje, contudo, parecem pouco explicativas, mas é seu conteúdo que interessa examinar.

\section{Conclusão}

Darcy, vê a universidade como veículo de transformação da sociedade, a qual, por sua vez, só pode ser transformada via soluções políticas. Logo, o papel da universidade, para Darcy, indissocia-se da criação de uma consciência crítica. A universidade é uma instituição social repleta de ideologias e interesses, politizada, com a missão de nortear o desenvolvimento autônomo de sua nação. A neutralidade das ciências é vista como falaciosa e é recusada por Darcy, o qual acredita que a despolitização da universidade resultaria na sua submissão aos interesses e à lógica dominante de distribuição de poder numa sociedade que não rompe com sua condição de atraso e de subdesenvolvimento. A transformação da sociedade exige a política. A universidade tem, pois, um papel político.

\begin{thebibliography}{99}
\bibitem{ref1} A. M. Ribeiro e G. R. Matias, \textit{A universidade necessária em Darcy Ribeiro: notas sobre  um pensamento utópico}. Unisinos, 2006.
\bibitem{ref2} A. M. Ribeiro, \textit{Darcy Ribeiro e UnB: intelectuais, projeto e missão}. Schielo, 2016.
\bibitem{ref3} M. T. D. Ribeiro, \textit{A PESQUISA E A UNIVERSIDADE}.
\end{thebibliography}

\end{document}
